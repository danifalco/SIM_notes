\section{Distributions}

Understanding different probability distributions is crucial in statistical modelling and inference. Each distribution has unique properties and applications, and choosing the correct one for your data is essential.

\subsection{Key Distributions and Their Properties}

\begin{itemize}
    \item \textbf{Normal Distribution}:
    \begin{itemize}
        \item Symmetrical, bell-shaped distribution defined by its mean \(\mu\) and variance \(\sigma^2\).
        \item Widely used in inferential statistics due to the Central Limit Theorem.
        \item \textbf{R Functions}: \texttt{dnorm()}, \texttt{pnorm()}, \texttt{rnorm()}.
    \end{itemize}
    \item \textbf{Binomial Distribution}:
    \begin{itemize}
        \item Models the number of successes in a fixed number of independent Bernoulli trials.
        \item Defined by parameters \(n\) (number of trials) and \(p\) (probability of success).
        \item \textbf{R Functions}: \texttt{dbinom()}, \texttt{pbinom()}, \texttt{rbinom()}.
    \end{itemize}
    \item \textbf{Poisson Distribution}:
    \begin{itemize}
        \item Models the count of events occurring in a fixed interval of time or space.
        \item Parameterized by \(\lambda\) (rate of occurrence).
        \item \textbf{R Functions}: \texttt{dpois()}, \texttt{ppois()}, \texttt{rpois()}.
    \end{itemize}
    \item \textbf{Exponential Distribution}:
    \begin{itemize}
        \item Models the time between successive events in a Poisson process.
        \item Defined by the rate parameter \(\lambda\).
        \item \textbf{R Functions}: \texttt{dexp()}, \texttt{pexp()}, \texttt{rexp()}.
    \end{itemize}
    \item \textbf{Chi-squared Distribution}:
    \begin{itemize}
        \item Distribution of the sum of squared standard normal variables.
        \item Commonly used in hypothesis testing and constructing confidence intervals.
        \item \textbf{R Functions}: \texttt{dchisq()}, \texttt{pchisq()}, \texttt{rchisq()}.
    \end{itemize}
    \item \textbf{Student’s t Distribution}:
    \begin{itemize}
        \item Similar to the normal distribution but with heavier tails.
        \item Useful for small sample sizes or when population variance is unknown.
        \item \textbf{R Functions}: \texttt{dt()}, \texttt{pt()}, \texttt{rt()}.
    \end{itemize}
\end{itemize}

\subsection{Choosing the Right Distribution}
The choice of distribution depends on the type of data and the problem context. Consider the following guidelines:

\begin{itemize}
    \item \textbf{Continuous Data}: Use the normal distribution if the data is symmetrically distributed. Use the exponential distribution for time-to-event data.
    \item \textbf{Discrete Data}: Use the binomial distribution for binary outcomes and the Poisson distribution for count data.
    \item \textbf{Small Sample Size}: Use the t-distribution if the sample size is small and the variance is unknown.
\end{itemize}

\subsection{Useful R Functions for Distributions}
\begin{itemize}
    \item \texttt{dnorm(), dpois(), dbinom()} --- Probability density functions.
    \item \texttt{pnorm(), ppois(), pbinom()} --- Cumulative distribution functions.
    \item \texttt{rnorm(), rpois(), rbinom()} --- Random number generation.
    \item \texttt{qnorm(), qpois(), qbinom()} --- Quantile functions.
\end{itemize}

\subsection{Example R Code for Distributions}
Here’s how to generate and visualize different distributions in R:

\begin{lstlisting}[language=R, caption=R Code for Generating and Visualizing Distributions]
# Set up parameters
mu <- 0
sigma <- 1
lambda <- 2
n <- 10
p <- 0.3

# Generate and visualize a normal distribution
x <- seq(-3, 3, by = 0.1)
y <- dnorm(x, mean = mu, sd = sigma)
plot(x, y, type = "l", main = "Normal Distribution", xlab = "X", ylab = "Density")

# Generate and plot a binomial distribution
binom_data <- dbinom(0:n, size = n, prob = p)
barplot(binom_data, main = "Binomial Distribution (n=10, p=0.3)", xlab = "Number of Successes")

# Generate and plot a Poisson distribution
pois_data <- dpois(0:20, lambda = lambda)
barplot(pois_data, main = "Poisson Distribution (lambda=2)", xlab = "Number of Events")

# Generate and plot an exponential distribution
exp_data <- rexp(1000, rate = lambda)
hist(exp_data, main = "Exponential Distribution (rate=2)", xlab = "Time", breaks = 20)
\end{lstlisting}

\subsection{When to Use Each Distribution}

\begin{tabularx}{\textwidth}{|l|X|X|X|}
\hline
\textbf{Distribution} & \textbf{Type of Data} & \textbf{Common Usage} & \textbf{R Function} \\
\hline
Normal & Continuous & Modeling measurement errors, heights, weights & \texttt{rnorm(), dnorm()} \\
\hline
Binomial & Discrete & Modeling binary outcomes (e.g., coin flips) & \texttt{rbinom(), dbinom()} \\
\hline
Poisson & Discrete & Modeling count data (e.g., number of calls per hour) & \texttt{rpois(), dpois()} \\
\hline
Exponential & Continuous & Modeling time to next event (e.g., failure times) & \texttt{rexp(), dexp()} \\
\hline
Chi-squared & Continuous & Goodness-of-fit tests & \texttt{rchisq(), dchisq()} \\
\hline
Student's t & Continuous & Small sample data, unknown variance & \texttt{rt(), dt()} \\
\hline
\end{tabularx}

\subsection{Key Points}
\begin{itemize}
    \item Understanding the properties of each distribution is essential for selecting the appropriate model.
    \item Use visualizations (e.g., histograms, density plots) to assess whether your data fits a particular distribution.
    \item Some statistical tests rely on specific distributions; always verify assumptions before proceeding.
\end{itemize}
