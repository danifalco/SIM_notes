\section{Exploratory Data Analysis}

Exploratory Data Analysis (EDA) is a crucial step in understanding and summarizing the main characteristics of a dataset. It involves the use of various statistical tools and visualization techniques to identify patterns, detect outliers, and gain insights before conducting formal modeling or hypothesis testing.

\subsection{Numerical Summaries}
\textbf{Numerical summaries} provide quantitative insights into the dataset using measures of central tendency and variability. Key statistics include:

\begin{itemize}
    \item \textbf{Mean} (\(\mu\)): The average of all observations in a dataset.
    \item \textbf{Median} (\(Q_2\)): The middle value of the ordered dataset, less sensitive to outliers.
    \item \textbf{Mode}: The most frequently occurring value(s) in the dataset.
    \item \textbf{Variance} (\(\sigma^2\)): Measure of how far observations spread out from the mean.
    \item \textbf{Standard Deviation} (\(\sigma\)): The square root of the variance.
    \item \textbf{Interquartile Range} (IQR): Difference between the 75th percentile (\(Q_3\)) and 25th percentile (\(Q_1\)).
\end{itemize}

\subsection{Graphical Summaries}
Graphical representations help to visualize the distribution and relationships between variables. Common plots include:

\begin{itemize}
    \item \textbf{Histograms}: Show the frequency distribution of a continuous variable.
    \item \textbf{Box Plots}: Display the spread of the data along with outliers.
    \item \textbf{Scatter Plots}: Used to visualize the relationship between two continuous variables.
    \item \textbf{Bar Charts}: Represent categorical data using bars of different heights.
\end{itemize}

\subsection{Useful R Functions for EDA}
\begin{itemize}
    \item \texttt{summary()} --- Provides a numerical summary of a dataset.
    \item \texttt{boxplot()} --- Plots a box-and-whisker plot.
    \item \texttt{hist()} --- Creates a histogram for a given variable.
    \item \texttt{plot()} --- Generic plotting function, often used for scatter plots.
\end{itemize}

\subsection{Example R Code for EDA}
Below is an example of performing basic EDA on a sample dataset in R:

\begin{lstlisting}[language=R, caption=Basic EDA in R]
# Load necessary library
library(ggplot2)

# Example dataset: mtcars
data("mtcars")

# Numerical summary
summary(mtcars)

# Histogram of Miles per Gallon (mpg)
hist(mtcars$mpg, main="Histogram of MPG", xlab="Miles Per Gallon", col="lightblue", breaks=10)

# Boxplot for distribution of mpg by number of cylinders
boxplot(mpg ~ cyl, data=mtcars, main="Boxplot of MPG by Cylinder", xlab="Number of Cylinders", ylab="Miles Per Gallon", col="orange")

# Scatter plot of Horsepower vs MPG
plot(mtcars$hp, mtcars$mpg, main="Scatter Plot of Horsepower vs MPG", xlab="Horsepower", ylab="Miles Per Gallon", col="blue", pch=19)
\end{lstlisting}

\subsection{Key Points}
\begin{itemize}
    \item EDA helps to identify anomalies and understand the structure of the data.
    \item Both numerical and graphical summaries are essential in the preliminary stages of data analysis.
    \item Visualizations are often more intuitive and revealing compared to raw statistical measures.
\end{itemize}
