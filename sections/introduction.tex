\section{Introduction}

This cheat sheet provides a concise summary of key concepts in statistical inference and modeling. It is structured to help navigate through definitions, usages, and code snippets for statistical tools commonly used in data science.

\subsection{Classical vs Fisherian Inference}
Statistical inference is broadly classified into two paradigms:
\begin{itemize}
    \item \textbf{Classical Inference (Frequentist Approach)}: Focuses on the long-run frequency of events. Relies heavily on hypothesis testing, confidence intervals, and p-values.
    \item \textbf{Fisherian Inference}: An approach that uses likelihood to measure support for different hypotheses. Unlike the classical approach, Fisher’s inference does not require specification of alternative hypotheses.
\end{itemize}

\subsection{Key Concepts in Inference}
Statistical inference involves making decisions or drawing conclusions about a population based on a sample. The two primary goals are:
\begin{enumerate}
    \item \textbf{Estimation}: Determining the approximate value of a population parameter.
    \item \textbf{Hypothesis Testing}: Assessing evidence provided by the data against a specified hypothesis.
\end{enumerate}

\subsection{When to Use Different Methods}
\begin{itemize}
    \item \textbf{Parametric Tests} (e.g., t-tests, ANOVA): Assume underlying distribution of the data.
    \item \textbf{Non-Parametric Tests} (e.g., Mann-Whitney U test, Kruskal-Wallis test): Do not assume any specific distribution.
\end{itemize}

\subsection{Useful R Functions for Inference}
\begin{itemize}
    \item \texttt{t.test()} --- Performs one and two-sample t-tests.
    \item \texttt{chisq.test()} --- Performs Chi-squared test for independence.
    \item \texttt{prop.test()} --- Tests for the equality of proportions.
\end{itemize}
