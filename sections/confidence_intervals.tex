\section{Confidence Intervals}

Confidence intervals provide a range of plausible values for an unknown population parameter based on sample data. They are used to quantify the uncertainty around a point estimate and are typically expressed as a percentage (e.g., 95\% confidence interval).

\subsection{Key Concepts in Confidence Intervals}
\begin{itemize}
    \item \textbf{Point Estimate}: A single value estimate of a population parameter (e.g., sample mean).
    \item \textbf{Confidence Level}: The proportion of times the interval would contain the true parameter if we were to repeat the sampling process many times (e.g., 95\%).
    \item \textbf{Margin of Error}: The maximum expected difference between the point estimate and the true population parameter. Influenced by sample size and variability.
    \item \textbf{Confidence Interval Formula}: 
        \[
        \text{CI} = \text{Point Estimate} \pm \text{Critical Value} \times \text{Standard Error}
        \]
\end{itemize}

\subsection{Confidence Interval for the Mean (Known Variance)}
When the population variance \(\sigma^2\) is known, the confidence interval for the mean \(\mu\) is calculated as:

\[
\text{CI} = \bar{x} \pm z_{\alpha/2} \times \frac{\sigma}{\sqrt{n}}
\]

Where:
\begin{itemize}
    \item \(\bar{x}\) --- Sample mean.
    \item \(z_{\alpha/2}\) --- Critical value from the standard normal distribution for the given confidence level.
    \item \(\sigma\) --- Population standard deviation.
    \item \(n\) --- Sample size.
\end{itemize}

\subsection{Confidence Interval for the Mean (Unknown Variance)}
When the population variance is unknown, use the sample standard deviation \(s\) and the t-distribution:

\[
\text{CI} = \bar{x} \pm t_{\alpha/2, n-1} \times \frac{s}{\sqrt{n}}
\]

Where \(t_{\alpha/2, n-1}\) is the critical value from the t-distribution with \(n-1\) degrees of freedom.

\subsection{Confidence Interval for Proportions}
For categorical data, the confidence interval for a proportion \(p\) is calculated using:

\[
\text{CI} = \hat{p} \pm z_{\alpha/2} \times \sqrt{\frac{\hat{p}(1-\hat{p})}{n}}
\]

Where:
\begin{itemize}
    \item \(\hat{p}\) --- Sample proportion.
    \item \(n\) --- Sample size.
    \item \(z_{\alpha/2}\) --- Critical value from the standard normal distribution.
\end{itemize}

\subsection{Useful R Functions for Confidence Intervals}
\begin{itemize}
    \item \texttt{t.test()} --- Provides confidence intervals for one-sample, two-sample, and paired t-tests.
    \item \texttt{prop.test()} --- Computes confidence intervals for proportions.
    \item \texttt{confint()} --- Calculates confidence intervals for model parameters.
\end{itemize}

\subsection{Example R Code for Confidence Intervals}
Here’s how to compute various types of confidence intervals in R:

\begin{lstlisting}[language=R, caption=Confidence Interval Calculations in R]
# One-sample t-test for confidence interval
x <- c(12.9, 14.2, 13.7, 15.3, 14.8, 13.6, 14.0)
t.test(x) # 95% CI for the mean

# Confidence interval for a proportion
prop.test(x = 34, n = 230, conf.level = 0.95) # 95% CI for proportion

# Confidence interval for a difference in means
group1 <- c(15, 17, 16, 14, 18)
group2 <- c(12, 14, 13, 15, 11)
t.test(group1, group2) # 95% CI for the difference in means

# Confidence intervals for regression coefficients
model <- lm(mpg ~ hp + wt, data = mtcars)
confint(model, level = 0.95) # 95% CI for model coefficients
\end{lstlisting}


\newcolumntype{Y}{>{\centering\arraybackslash}X}
\subsection{Choosing the Appropriate Confidence Interval}
\begin{tabularx}{\textwidth}{|l|Y|X|X|}
\hline
\textbf{Scenario} & \textbf{Parameter of Interest} & \textbf{Distribution} & \textbf{R Function} \\
\hline
Population mean with known variance & \(\mu\) & Normal distribution & Manual calculation or \texttt{qnorm()} \\
\hline
Population mean with unknown variance & \(\mu\) & t-distribution & \texttt{t.test()} \\
\hline
Population proportion & \(p\) & Binomial distribution & \texttt{prop.test()} \\
\hline
Difference in two population means & \(\mu_1 - \mu_2\) & t-distribution & \texttt{t.test()} \\
\hline
Regression coefficients & \(\beta\) & Normal approximation & \texttt{confint()} \\
\hline
\end{tabularx}

\subsection{Key Points}
\begin{itemize}
    \item Higher confidence levels (e.g., 99\%) produce wider intervals compared to lower confidence levels (e.g., 90\%).
    \item Confidence intervals are not probabilities; they describe the range of plausible values for the population parameter.
    \item If a confidence interval for a mean difference contains zero, it suggests no significant difference.
\end{itemize}
